\begin{recomendations}
    \par Repasando los puntos que pueden mejorarse del algoritmo propuesto, se encuentra encontrar una forma eficiente y efectiva de resolver el sistema de ecuaciones que permita considerar patrones con dimensiones mucho mayores a la actual. Otro punto es la sensibilidad al ruido que presenta el algoritmo, pues usar solo una porci\'on de la anomal\'ia puede desembocar en un gran n\'umero falsos positivos.
    \par Un aspecto que no se mencion\'o pero que cabe puntualizar para futuras investigaciones sobre el tema es la posiblidad de usar patrones que aparecen en distinta proporci\'on en la imagen a analizar. En la presente investigaci\'on las im\'agenes que se usaron de experimento eran im\'agenes de tama\~no real reducidas con la DWT un n\'umero de veces igual al del patr\'on, por lo que siempre se encontraban en la misma proporci\'on en los casos en los que aparec\'ia. Es necesario tener en cuenta que un patr\'on en distinta proporci\'on significa informaci\'on extra o faltante, por lo que el proceso de detecci\'on podr\'ia tornarse m\'as complejo.
\end{recomendations}
