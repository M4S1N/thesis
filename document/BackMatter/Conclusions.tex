\begin{conclusions}
    \par Se propuso un m\'etodo para la detecci\'on de anomal\'ias en mamograf\'ias digitales usando la transformada de Shapelet en dos dimensiones. Para ello se brindaron un conjunto de herramientas y resultados que pueden ser empleados en futuros estudios de la teor\'ia wavelets y no solo para el reconocimiento en im\'agenes. El algoritmo propuesto aunque presenta un buen n\'umero de fallos a la hora de tratar con im\'agenes que contienen el patr\'on con ruido, raz\'on por la cual se us\'o la alternativa de seleccionar una porci\'on de la anomal\'ia para los experimentos finales, posee una gran efectividad a la hora de reconocer patrones bien definidos como fue el caso de detectar una letra en un conjunto de palabras, resultado que indica que, aunque quedan muchos aspectos por mejorar el uso de shapelets para el reconocimiento de patrones, es un camino correcto.
    \par Cabe destacar tambi\'en los resultados obtenidos durante la experimentaci\'on en una dimensi\'on y el an\'alisis de los m\'etodos de resoluci\'on de sistemas de ecuaciones, pues permitieron sentar las bases para el desarrollo del concepto de Shapelet bidimensional.
    \par De manera general, se cumpli\'o el objetivo principal que era construir una base wavelet que permitiera la detecci\'on de anomal\'ias en mamograf\'ias y ,se reitera, aunque quedan varios aspectos que mejorar, se considera que este m\'etodo pueda ser utilizado en un futuro en la medicina para detectar el c\'ancer de mama en una etapa temprana.
\end{conclusions}
