\begin{opinion}
    \par La transformada wavelet se ha convertido en una de las técnicas más utilizadas para analizar las señales de audio e imágenes. Esta transformada se basa en funciones matemáticas especiales llamadas wavelets. La construcción de funciones wavelets es siempre un tema interesante y de gran aplicabilidad. Este es precisamente el tópico de esta tesis.
    
\par En la literatura se reportan muchos enfoques de construcción que optimizan los parámetros matemáticos de dichas funciones como la regularidad, diferenciabilidad, momentos nulos, entre otras. Los métodos que permiten construir wavelets que sean capaces de reconocer en una señal determinados patrones definidos de antemano es menos frecuente. La creación de estas wavelets para patrones en 2 dimensiones (imágenes) no es tan frecuente en la literatura al respecto.

\par La investigación realizada por el estudiante Miguel Alejandro Asin Barthelemy se basa en la Transformada Discreta de Daubechies y propone una alternativa de extensión 2D para dicha transformada que permite detectar patrones en 2D. Para ello, primeramente, implementó la transformada unidimensional y realizó experimentos para validar su eficiencia en la detección, así como la solución numérica del sistema de ecuaciones no lineales subyacente. Se validó la propuesta en señales artificiales y en imágenes de mamografía digital. 

\par Durante el desarrollo de la tesis Miguel estudió la literatura referente al análisis wavelet teórico de forma seria y crítica, proponiendo formas de cómputo de ciertas propiedades y parámetros involucrados en los algoritmos. Además, evaluó sus algoritmos en varias configuraciones experimentales, mostrando las ventajas y desventajas de este enfoque respecto a las wavelets clásicas. 

\par Para esta tesis Miguel estudió las materias referidas, incluidas parcialmente en el currículo de la carrera, mostró disciplina, entrega y rigor. Además, demostró habilidades para el trabajo con la bibliografía y creatividad para proponer soluciones a problemas teórico-computacionales y de implementación, entre otras competencias de programación en el lenguaje Python y sus diversos frameworks. 

\par La alternativa 2D propuesta mostró éxito en mamografía, por lo que considero que el estudiante logró cumplir exitosamente el objetivo propuesto y mostró que ciertos caminos no son adecuados para la extensión 2D de la transformada, valor agregado de esta tesis.

\par Por tanto, considero que a esta tesis del estudiante Miguel Alejandro Asin Barthelemy debe otorgársele la máxima calificación (5 puntos, Excelente), y estoy seguro que en el futuro se desempeñará como un excelente profesional de la Ciencia de la Computación.

\begin{flushright}
\begin{figure}[h]
\flushright
\includegraphics[scale=.1]{Graphics/tutorfirma.jpg}
\end{figure}
MSc. Damian Valdés Santiago\\
26 de noviembre de 2022
\end{flushright}
\end{opinion}