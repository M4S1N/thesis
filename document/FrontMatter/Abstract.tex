\begin{resumen}
	\par Se propone el desarrollo de un algoritmo para determinar una base wavelet que permita la detección de anomalías en mamografías digitales, para ello se definen un conjunto de conceptos pertenecientes a la teoría wavelet que serán usados para introducir un nuevo tipo llamado shapelet. Las shapelets son wavelet que, a diferencia de otras, tienen en cuenta la forma de la señal que está analizando. Se muestra un algoritmo para crear una base shapelet a partir de un patrón, la cual es usada para detectar dicho patrón en una señal de muestra. Para ello, se construye un sistema de ecuaciones no lineales cuya solución es el vector que se usará de base. Con estos conceptos y una introducción a las wavelets en dos dimensiones, se propone una forma de extender las shapelets al análisis de imágenes. Posteriormente, se adapta el sistema de ecuaciones a dos dimensiones teniendo en cuenta las traslaciones diádicas que se producen durante el cálculo de la transformada y, tras resolver el sistema, se obtiene nuevamente la base shapelet. Como experimento se toma una colección de mamografías de las cuales algunas presentan algún tipo de anomalía y se le aplica el algoritmo propuesto, evaluando de esta forma el modelo creado a partir de los resultados obtenidos.
\end{resumen}

\begin{abstract}
	The development of an algorithm to determine a wavelet base that allows the detection of anomalies in digital mammograms is proposed, for which a set of concepts belonging to wavelet theory are defined that will be used to introduce a new type called shapelet. Shapelets are wavelets that, unlike other wavelets, take into account the shape of the signal you are analyzing. An algorithm is shown to create a shapelet base from a pattern, which is used to detect said pattern in a sample signal. To do this, a system of nonlinear equations is built whose solution is the vector that will be used as the base. With these concepts and an introduction to wavelets in two dimensions, a way to extend shapelets to image analysis is proposed. Subsequently, the system of equations is adapted to two dimensions taking into account the dyadic translations that occur during the calculation of the transform and, after solving the system, the shapelet basis is obtained again. As an experiment, a collection of mammograms is taken, some of which present some type of anomaly and the proposed algorithm is applied, thus evaluating the model created from the results obtained.
\end{abstract}