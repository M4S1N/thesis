\begin{resumen}
	\par Se propone el desarrollo de un algoritmo para determinar una base wavelet que permita la detección de anomalías en mamografías digitales, para ello definen un conjunto de conceptos pertenecientes a la teoría wavelet que serán usados para introducir un nuevo tipo llamado Shapelet. Las Shapelet son wavelet que a diferencia de otras wavelets tienen en cuenta la forma de la señal que está analizando. Se muestra un algoritmo para crear una base Shapelet a partir de un patrón y usada para detectar dicho patrón en una señal de muestra, para ello se construye un sistema de ecuaciones no lineal cuya solución la constituye el vector que se usará de base. Con estos conceptos y una introducción a las wavelets en dos dimensiones se propone una forma de extender las Shapelet al análisis de imágenes. Posteriormente se adapta el sistema de ecuaciones a dos dimensiones teniendo en cuentra las traslaciones diádicas que se producen durante el cálculo de la transformada y, tras resolver el sistema se obtiene nuevamente la base Shapelet. Como experimento se toma una colección de mamografías de las cuales algunas presentan algún tipo de anomalía y se le aplica el algoritmo propuesto. Evaluando de esta forma el modelo creado a partir de los resultados obtenidos.
\end{resumen}

\begin{abstract}
	The development of an algorithm to determine a wavelet base that allows the detection of anomalies in digital mammograms is proposed, for this they define a set of concepts belonging to the wavelet theory that will be used to introduce a new type called Shapelet. Shapelets are wavelets that, unlike other wavelets, take into account the shape of the signal being analyzed. An algorithm is shown to create a Shapelet base from a pattern and used to detect said pattern in a sample signal, for this a non-linear system of equations is built whose solution is the vector that will be used as the base. With these concepts and an introduction to wavelets in two dimensions, a way to extend Shapelets to image analysis is proposed. Subsequently, the system of equations is adapted to two dimensions taking into account the dyadic translations that occur during the calculation of the transform and, after solving the system, the Shapelet basis is obtained again. As an experiment, a collection of mammograms is taken, some of which present some type of anomaly, and the proposed algorithm is applied. Evaluating in this way the model created from the results obtained.
\end{abstract}