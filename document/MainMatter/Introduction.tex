\chapter*{Introducción}\label{chapter:introduction}
\addcontentsline{toc}{chapter}{Introducción}

\par Entre las principales causas de muerte por c\'ancer en el mundo, el c\'ancer de mama ocupa uno de los primeros lugares entre las mujeres. Seg\'un datos proporcionados por la Organizaci\'on Mundial de la Salud, en el 2020, el c\'ancer de mama ocup\'o el primer lugar en cantidad de casos con $2.26$ millones y el quinto en defunciones con 685 mil muertes [\cite{16}]. En el caso de Am\'erica, el c\'ancer de mama constituye la principal causa de muerte por c\'ancer, dadas cifras tambi\'en del 2020, pues ese a\~no se diagnosticaron 210 mil nuevos casos y casi 68 mil defunciones.

\par Seg\'un los especialistas el cáncer de mama se presenta con mayor frecuencia como una masa indolora en la mama, que puede desarrollarse por razones distintas al cáncer. Otros síntomas del cáncer de mama incluyen engrosamiento de la mama, alteración de su tamaño, la forma o la apariencia de la mama, alteraciones de la piel como enrojecimiento, picaduras o hoyuelos, cambio en la apariencia del pezón o la piel alrededor (areola), y/o secreción anormal del pezón [\cite{17}].

\par Sin embargo, el tratamiento del c\'ancer de mama puede ser efectivo en caso de detectarse a tiempo, mediante el uso de mamograf\'ias. Estas mamograf\'ias son analizadas por un radi\'ologo que se encarga de dar una clasificaci\'on (de 0 a 6) basado en un sistema est\'andar llamado \textit{Breast Imaging Reporting and Data System} o BI-RADS [\cite{19}]. En varios casos es posible calificar una radiograf\'ia con categor\'ia cero, esto se debe a que el radi\'ologo pudo haber visto una posible anomalía, pero que no está definida con claridad y que podr\'ian necesitarse exámenes adicionales. En dependencia de los recursos de los cuales se dispone, este hecho puede representar un problema, pues no podr\'ia ser posible la creaci\'on de nuevas mamograf\'ias y, en caso de ser positiva la muestra al c\'ancer de mama, podr\'ia desembocar en un agravamiento de la condici\'on.

\par Teniendo en cuenta este problema, se han buscado formas de obtener una calificaci\'on m\'as acertada mediante el uso de equipos de c\'omputo, la cual podr\'ia constituir una v\'ia factible teniendo en cuenta el desarrollo actual en el campo del procesamiento de im\'agenes digitales y la detecci\'on de patrones.

\par En este sentido, se han creado un gran n\'umero de algoritmos para el reconocimiento de patrones [\cite{20}], con todo tipo de aplicaciones en la sociedad actual, las cuales van desde detecci\'on de rostros en fotograf\'ias hasta el an\'alisis y reconocimiento de estrellas y galaxias. Haciendo usos de t\'ecnicas de inteligencia artificial y suficientes datos de muestra, ha sido posible entrenar varios modelos de clasificaci\'on que permiten la detecci\'on efectiva de las caracter\'isticas para las cuales fue desarrollado.

\par Un segundo m\'etodo consiste en el uso de wavelets para la detecci\'on de patrones. Las wavelets son funciones con la peculiaridad de que son muy efectivas a la hora de analizar una se\~nal en tiempo y frecuencia. Si se tiene en cuenta que una imagen puede ser interpretada como una se\~nal bidimensional, es posible ampliar el concepto de las wavelets al an\'alisis de im\'agenes o lo que es, en este caso, mamograf\'ias digitales.

\par Se desea entonces crear una base de funciones wavelets que permita analizar un patr\'on de una anomal\'ia o tumor cancer\'igeno y que sea capaz de detectarlo en mamograf\'ias digitales de muestra. Para ello, ser\'a necesario entender el concepto de wavelet, c\'omo estas podr\'ian emplearse en la detecci\'on de patrones unidimensionales y, posteriormente, extender estos conceptos a dos dimensiones para tratar con im\'agenes o mamograf\'ias.

\par A continuaci\'on se brindar\'a el panorama en el cual se gest\'o la idea del uso de las funciones wavelet para el reconocimiento de patrones. Se propondr\'a un tipo de wavelet que se define espec\'ificamente para reconocimiento de patrones; caracter\'istica que resultar\'a muy \'util en la tarea de detectar anomal\'ias en im\'agenes. El método usado para construir dicha wavelet resuelve un sistema de ecuaciones no lineales. Por ello, en una segunda instancia, se analizar\'a un conjunto de m\'etodos utilizados para la resoluci\'on de sistemas de ecuaciones no lineales con el objetivo de encontrar el que mejor compatibilidad tenga con el modelo que se usar\'a y, posteriormente, se emplear\'an las herramientas adquiridas para desarrollar una extensi\'on hacia dos dimensiones del tipo de wavelet definida en una dimensi\'on y un algoritmo que permita la detecci\'on del patr\'on.

\par Por \'ultimo, se presentar\'a un \textit{software} que lleve a la pr\'actica los conceptos que se analizar\'an, tanto para una como para dos dimensiones, en correspondencia con los resultados obtenidos por el programa para distintos casos que podr\'ian presentarse durante la tarea de reconocimiento. Se dar\'a una evaluaci\'on del m\'etodo propuesto y recomendaciones para alcanzar una mejor precisi\'on, resultado que podr\'ia culminar con la creaci\'on de una herramienta que pueda ser utilizada por radi\'ologos para la detecci\'on temprana de esta peligrosa enfermedad.\\

\par El presente trabajo pertenece a acciones del proyecto ``Métodos numéricos para problemas en múltiples escalas''. Proyecto asociado al Programa Nacional de Ciencias Básicas, Código PN223LH010-003, Ministerio de Ciencia, Tecnología y Medio Ambiente (CITMA), Cuba, 2021-2023.